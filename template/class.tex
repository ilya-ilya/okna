\documentclass[a4paper, 11pt]{article}

\usepackage[utf8]{inputenc}
\usepackage[russian]{babel}
%\usepackage{pscyr}
\usepackage{mathtools}
\usepackage{amsmath, amssymb, amsfonts}
\usepackage{graphicx}
\usepackage[update]{epstopdf}
\usepackage[nosimple,dots,thmitshape,nodiagram]{dmvn}
\usepackage[all]{xy}

\textheight = 23.5 cm
\textwidth = 16.5 cm
\hoffset = -2cm
\voffset = -2cm
\tolerance = 4000
\begin{document}
% good reference for equations
\renewcommand{\theequation}{\thesection.\arabic{equation}}
% At least, because ex counter resets at start of section
\renewcommand{\theex}{\thesection.\arabic{ex}}
% better equal by definition
\renewcommand{\eqdef}{\triangleq}

% Transoceanization
\renewcommand{\emptyset}{\varnothing}
\renewcommand{\phi}{\varphi}
\renewcommand{\epsilon}{\varepsilon}

\newcommand{\Cx}{\mathbb{C}}
\newcommand{\Hx}{\mathbb{H}}
\newcommand{\Zm}[1]{\mathbb{Z}_{#1}}
\newcommand{\fA}{~\forall\;}
\newcommand{\Ex}{~\exists\;}
\newcommand{\Exo}{~\exists\,!\;}
\title{Введение}
\date{19 октября 2016 г.}
\author{}
\maketitle{}
\section{Графы}
\begin{df}
	Пусть задано некоторое конечное множество объектов или элементов, некоторые из
	которых попарно связаны между собой. Тогда данное множество элементов, а также весь набор связей
	между этими элементами называются \textit{графом} $G$. В этом случае данные объекты или элементы называются
	\textit{вершинами графа}, а связи между ними — \textit{рёбрами графа}. Вершины, связанные ребром, называются
	концами этого ребра. Такие вершины называются \textit{смежными}.
\end{df}
\begin{df}
	\textit{Степенью вершины} называется количество рёбер, выходящих из этой вершины.
	Если это количество чётно, то вершина называется чётной, в противном случае вершина называется
	нечётной.
\end{df}
\begin{denotes}
	$G = G(V, E)$, где $V$ -- множество вершин, а $E$ -- множество рёбер.

	Если $v_1, v_2 \in V$ -- вершины, то $(v_1, v_2) \in E$ -- ребро их соединяющее.

	Если $v \in V$ -- вершина графа $G$, то $\deg_G (v)$ -- её степень.
\end{denotes}
\begin{problem}
	Пусть в графе $n$ вершин и каждая соединена с каждой. Сколько в таком графе рёбер?
\end{problem}
\begin{solution}
	Давайте посчитаем концы рёбер.
	По условию в каждой вершине сходится $n - 1$ ребро, значит всего концов рёбер $n \cdot (n - 1)$.
	Но заметим, что у каждого ребра ровно два конца, значит рёбер в два раза меньше, чем концов.
	Следовательно их $\frac{n \cdot (n - 1)}2$.
\end{solution}
\begin{answer}
	$\frac{n \cdot (n - 1)}2$
\end{answer}
\begin{note}
	Это решение, более формальное и сухое, предпочтительнее того, которое рассказывалось на занятии.
\end{note}
\begin{df}
	Граф из предыдущей задачи называется \textit{полным}.
\end{df}
\section{Комбинаторика}
\begin{problem}
	Пусть имеется $n$ карандашей разного цвета и $n$ детей\footnote{Пояснение на случай,
	если не совсем ясно: дети считаются различными}. Сколькими способами можно раздать
	карандаши детям?
\end{problem}
\begin{answer}
	$n!$
\end{answer}
\begin{problem}
	Автомобильные номера в одном регионе РФ состоят либо из 3 букв и 3 цифр, либо из 2 букв и 4
	цифр, при этом порядок следования букв и цифр в номере фиксирован. Из букв используются не все, а
	только а, в, е, к, м, н, о, р, с, т, у, х. Какое максимальное число автомобилей может быть в одном регионе?
\end{problem}
\begin{answer}
	$12^3 \cdot 10^3 + 12^2 \cdot 10^4$
\end{answer}
\section{Бином Ньютона и треугольник Паскаля}
\begin{df}
	Рассмотрим числа $C_n^i$ ($i$ -- верхний индекс -- номер элемента, $n$ -- нижний индекс -- номер строки),
	\sth $C_{n - 1}^i + C_{n -1}^{i - 1} = C_n^i$, $C_n^0 \bw = C_n^n \bw = 1$, $C_0^0 = 1$.
\end{df}
\begin{note}
	Эти числа называются биномиальными коэффициентами и в скором времени у нас будет отдельный сюжет,
	посвящённый их свойствам.
	Но пока вполне хватит и определения.
\end{note}
\begin{problem}
	\begin{enumerate}
		\item Докажите, что
			$$
				(a + b)^n = \sum_{i = 0}^n C_n^i a^i b^{n-i};
			$$
		\item Найдите сумму всех элементов $n$\h ой строки треугольника Паскаля.
	\end{enumerate}
\end{problem}
\begin{solution}
\begin{enumerate}
	\item Будем доказывать методом математической индукции.

		База. При $n = 0$ получаем, что:
		$$
			(a + b)^0 = 1 = \sum_{i = 0}^0 1 \cdot 1 \cdot 1
		$$

		Шаг. Пусть для $n = k$ утверждение верно, \ie
		$$
			(a + b)^k = \sum_{i = 0}^k C_k^i a^i b^{k-i}.
		$$
		Тогда:
		\begin{gather*}
			(a + b) ^ {k+1} =
			(a + b) \cdot (a + b)^k \eqvl{пред. инд.}{30}
			(a + b) \cdot \hr{\sum_{i = 0}^k C_k^i a^i b^{k - i}} = \\
			= a \cdot \hr{\sum_{i = 0}^k C_k^i a^i b^{k - i}} + b \cdot \hr{\sum_{i = 0}^k C_k^i a^i b^{k - i}} = 
			\sum_{i = 0}^k C_k^i \cdot a \cdot a^i \cdot b^{k - i} +
				{\sum_{i = 0}^k C_k^i \cdot a^i \cdot b \cdot b^{k - i}} = \\
			= C_k^k a^{k + 1} + 
				\sum_{\substack{i = 1}}^{k} C_k^{i' - 1} \cdot a^{i'} \cdot b^{k + 1 - i'} +
				C_k^0 b^{k + 1} +
				\sum_{i = 1}^{k} C_k^{i} \cdot a^{i} \cdot b^{k + 1 - i} = \\
			= C_{k+1}^0 a^0 b^{k + 1} + 
				\sum_{i = 1}^{k} \hr{C_k^{i} + C_k^{i - 1}}\cdot a^{i} \cdot b^{k + 1 - i}
				+ C_{k+1}^{k+1} a^{k+1} b^0 = \\
			= \sum_{i = 0}^{k + 1} C_{k+1}^i a^i b^{k+1-i}
		\end{gather*}
	\item Одна из домашних задачек.
\end{enumerate}
\end{solution}
\end{document}
