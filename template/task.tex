\documentclass[a4paper, 11pt]{article}

\usepackage[utf8]{inputenc}
\usepackage[russian]{babel}
%\usepackage{pscyr}
\usepackage{mathtools}
\usepackage{amsmath, amssymb, amsfonts}
\usepackage{graphicx}
\usepackage[update]{epstopdf}
\usepackage[nosimple,dots,thmitshape,nodiagram]{dmvn}
\usepackage[all]{xy}

\textheight = 23.5 cm
\textwidth = 16.5 cm
\hoffset = -2cm
\voffset = -2cm
\tolerance = 4000
\begin{document}
% good reference for equations
\renewcommand{\theequation}{\thesection.\arabic{equation}}
% At least, because ex counter resets at start of section
\renewcommand{\theex}{\thesection.\arabic{ex}}
% better equal by definition
\renewcommand{\eqdef}{\triangleq}

% Transoceanization
\renewcommand{\emptyset}{\varnothing}
\renewcommand{\phi}{\varphi}
\renewcommand{\epsilon}{\varepsilon}

\newcommand{\Cx}{\mathbb{C}}
\newcommand{\Hx}{\mathbb{H}}
\newcommand{\Zm}[1]{\mathbb{Z}_{#1}}
\newcommand{\fA}{~\forall\;}
\newcommand{\Ex}{~\exists\;}
\newcommand{\Exo}{~\exists\,!\;}
\title{Введение. Домашка}
\date{19 октября 2016 г.}
\author{}
\maketitle{}
\section{Домашнее задание}
\begin{problem}
	Как и почему можно определить $0^0$?
\end{problem}
\begin{problem}
	Не производя вычислений найдите $6! \cdot 56$.
\end{problem}
\begin{problem}
	Существует ли граф, содержащий более одной вершины, никакие две вершины которого не имеют
	одинаковой степени?
\end{problem}
\begin{problem}
	Меню в школьном буфете постоянно и состоит из $n$ разных блюд.
	Петя хочет каждый день выбирать себе завтрак по-новому (за раз он может съесть от 0 до $n$ разных блюд).
	а) Сколько дней ему удастся это делать?
	б) Сколько блюд он съест за это время?
\end{problem}
\begin{problem}
	Посчитайте сумму элементов $n$\h ой строки треугоьника Пскаля.
\end{problem}
\end{document}
